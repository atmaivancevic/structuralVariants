\documentclass[a4paper,11pt]{article} % Default font size and paper size

\usepackage[T1]{fontenc}
\usepackage{fullpage}
\usepackage{enumitem}
\usepackage{amsmath, amssymb}
\usepackage[normalem]{ulem}
\usepackage{graphicx}
\usepackage{lipsum}
% Add quotes
\usepackage{csquotes}
\usepackage{color,hyperref}
\hypersetup{colorlinks=true,urlcolor=blue}

\definecolor{cornflowerblue}{rgb}{0.39,0.58,0.93}

\renewcommand{\thesection}{\arabic{section}}
%\usepackage{titlesec} % Used to customize the \section command
%\titleformat{\section}{\large\raggedright\bfseries}{}{0em}{} % Text formatting of sections
%\titleformat{\subsection}{\large\raggedright\bfseries}{}{0em}{} % Text formatting of sections
%\titlespacing{\section}{0pt}{10pt}{7pt} % Spacing around sections
%\def\labelitemi{--}

% Add bulltes
\usepackage{outlines}

%\pagestyle{empty} % Removes page numbering

\usepackage{geometry}
\geometry{
	a4paper,
	total={170mm,277mm},
	left=20mm,
	top=20mm,
	bottom=30mm,
	right=20mm,
}

\newenvironment{tightcenter}{%
	\setlength\topsep{0pt}
	\setlength\parskip{0pt}
	\begin{center}
	}{%
\end{center}
}
% Add url

\tolerance=1
\emergencystretch=\maxdimen
\hyphenpenalty=10000
\hbadness=10000

\begin{document} 


\noindent \textbf{\Large \title*{Structural Variant Calling for Whole Genome Sequences}}
\\

\noindent Atma Ivancevic \href{mailto:atma.ivancevic@adelaide.edu.au}{atma.ivancevic@adelaide.edu.au} \\
\textbf{Date created:} March 28, 2018 \\
\textbf{Date updated:} \today \\

\noindent \textbf{\large{\color{red}{WARNING: This pipeline is a work in progress. Some things may break as things change. Please let me know if this happens.}}}

\section*{Protocol Requirements:}
	\begin{outline}
		\1 This protocol is designed for use on a high-performance computer with SLURM queuing (e.g. \href{https://wiki.adelaide.edu.au/hpc}{wiki.adelaide.edu.au/hpc}).
		Scripts can be modified for use on a sufficiently powerful local computer.

		\1 You will need WGS bam files and the human reference genome.
			\2 We currently use hg19.
	\end{outline}

\section*{Step 1: Set up your files and work environment}
	\begin{outline}
		\1 This pipeline assumes all of your BAMs are in the same directory. If this is not the case, consider creating a new directory with symlinks to your BAMs. 
		E.g.:
		\begin{verbatim}
			mkdir /data/genomes
			cd /data/genomes
			ln -s /path/to/bam link_name
		\end{verbatim}

		\1 All BAM files need to be indexed. 
		This can be done with \href{http://www.htslib.org/}{SAMtools}.
		\begin{verbatim}
			samtools index bam_file
		\end{verbatim}
		
		\textit{\textbf{\color{cornflowerblue}{To quickly index many BAMs, try using indexBAM.sh.}}}
		
		\1 Set up folders for your output and log files.
		You can do this however you like. 
		
		This is how I set mine up:
		
		
		\begin{verbatim}
			# General output dir:
			$FASTDIR/outputs
		
			# Output dir for SV calling:
			$FASTDIR/outputs/SVcalling
				
			# Output dirs for each SV caller:
			$FASTDIR/outputs/SVcalling/dellyOut
			$FASTDIR/outputs/SVcalling/lumpyOut
			$FASTDIR/outputs/SVcalling/mantaOut
				
			# Dir for SLURM output:
			$FASTDIR/slurmOUT
			# Dir for SLURM logs: 
			$FASTDIR/slurmLOG
		\end{verbatim}
		
		\1 Download and install SV callers.
		Best practice is to use the latest stable version of each tool. 
		You can put executables in your PATH, or redirect to a dedicated ``executables'' directory. 
		
		\2 \href{https://github.com/dellytools/delly}{Delly}
		\2 \href{https://github.com/arq5x/lumpy-sv}{Lumpy}
		\2 \href{https://github.com/Illumina/manta}{Manta}
		
	\end{outline}

\section*{Step 2: Set off SV calling phase for each caller}
	\begin{outline}
	
	\1 Delly first. An example command for SV calling is:
	\begin{verbatim}
		delly call -g hg19.fa -o s1.bcf -x hg19.excl sample.bam
	\end{verbatim}
	
	\textit{\textbf{\color{cornflowerblue}{Use dellyCalling.sh to run on many genomes.
	You will need the reference genome (e.g. ucsc.hg19.fasta) and tsv file of regions to exclude (ucsc.hg19.excl.tsv).}}}

	
	\end{outline}

\end{document}